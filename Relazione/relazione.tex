\documentclass[]{article}

%opening
\usepackage{graphbox}	% permette di modificare i margini
\usepackage{graphicx}
\usepackage{float}
\usepackage{longtable}
\usepackage{array}
\usepackage{fancyhdr}
\pagestyle{fancy}

\newcommand{\copertina}{
	\begin{titlepage}
		\begin{center}
			
			\vspace{1cm}
			
			\begin{Huge}
				\textbf{Centro Estetico Nirvana} \\
			\end{Huge}
			
			\vspace{9pt}  
			
			\begin{large}
				\textbf{Progetto per il corso di Tecnologie Web\\}
				\textbf{A.A.} 2022/2023\\
				\vspace{3pt}
			\end{large}	  
			
			\vspace{24pt}
			
			\begin{large}
				\textbf{Indirizzo sito web:} \emph{todo}\\
			\end{large} 
			
			\vspace{10pt} 
			
			\bgroup
			\def\arraystretch{1.3}
			\centering
			\begin{tabular}{r|L{5cm}}
			\multicolumn{2}{c}{\textbf{Informazioni sul gruppo} } \\ \hline
			\textbf{Membri} &  Nicola Baesso - 2011877 \newline Matteo Cusin - 2008073 \newline Annalisa Egidi - 1216745 \newline Lisien Skenderi - 2023461\\
			\end{tabular}
			\egroup
			
			\vspace*{\fill}
			
			\begin{center}
				\textbf{Referente\\}
				Nicola Baesso - nicola.baesso@studenti.unipd.it\\
				\textbf{Utente\\}
				\textit{Amministratore:} admin - admin\\
				\textit{Cliente:} user - user\\
			\end{center}
			
		\end{center}
	\end{titlepage}
}	%FINE NEW COMMAND COPERTINA

\usepackage{lastpage}
\usepackage{fancyhdr}
\fancypagestyle{plain}{
	% cancella tutti i campi di intestazione e piè di pagina
	\fancyhf{}
	
	\lhead{
		Centro Estetico Nirvana
	}
	
	\lfoot{ %piè di pagina a sx
		Relazione Progetto Tecnologie Web
	}
	\rfoot{Pagina \thepage{} di \pageref{LastPage}} %es: pag: 4 di 10
	
	%linea orizzontale alle posizioni top e bottom della pagina (se è 0, non c'è la linea)
	\renewcommand{\headrulewidth}{0.3pt}  
	\renewcommand{\footrulewidth}{0.3pt}
}
\pagestyle{plain}

%\usepackage{calc} %introduce la notazione infissa per le op. aritmetiche interne a LaTeX

\usepackage[utf8]{inputenc}
%\usepackage{cm-super}

\usepackage{lmodern}
\usepackage[T1]{fontenc}
\usepackage[italian]{babel} %il documento è in italiano
%\usepackage{textcomp} %The pack­age sup­ports the Text Com­pan­ion fonts, which pro­vide many text sym­bols
%(such as baht, bul­let, copy­right, mu­si­cal­note, onequar­ter, sec­tion, and yen), in the TS1 en­cod­ing.

\usepackage{graphicx}       %permette di inserire delle immagini
\usepackage{caption}        %numerazione figure e loro descrizione testuale
\usepackage{subcaption}     %sottofigure numerabili
\usepackage{float}  %permette di inserire un # qualsiasi di figure fluttuanti
\usepackage[dvipsnames,table]{xcolor}
\usepackage{rotating} %permette di ruotare le immagini
%\usepackage{changepage} %utile se c'è bisogno di aggiustare margini per centrare figure

\usepackage{listings} %permette di inserire degli spezzoni di codice

\usepackage{tikz} %disegno di immagini vettoriali a schermo. Utile per grafi
\usetikzlibrary{arrows.meta}
\usetikzlibrary{graphs}
\usetikzlibrary{arrows}
%\usepackage{tikz-uml} %serve per disgnare l'UML, fantastica guida:
%https://perso.ensta-paristech.fr/~kielbasi/tikzuml/var/files/doc/tikzumlmanual.pdf
%download package: http://perso.ensta-paristech.fr/~kielbasi/tikzuml/

%package per le tabelle
\usepackage{booktabs} %permette di poter usare delle liste nelle tabelle
\usepackage{tabularx} 
\usepackage{longtable} %una tabella può continuare su più pagine
\usepackage{multirow} %utile per visualizzare una cella su più righe
%\usepackage{multicolumn} %cella su più colonne
%\usepackage[table]{xcolor} %rende disponibile l'utilizzo di un colore per lo sfondo
%delle celle di una tabella

%crea una cella per le tabelle in grado di andare a capo con \newline
%https://tex.stackexchange.com/questions/12703/how-to-create-fixed-width-table-columns-with-text-raggedright-centered-raggedlef
\usepackage{array}
\newcolumntype{L}[1]{>{\raggedright\let\newline\\\arraybackslash\hspace{0pt}}m{#1}}
\newcolumntype{C}[1]{>{\centering\let\newline\\\arraybackslash\hspace{0pt}}m{#1}}
\newcolumntype{R}[1]{>{\raggedleft\let\newline\\\arraybackslash\hspace{0pt}}m{#1}}


%indice con i puntini
\usepackage{tocloft}
\renewcommand\cftsecleader{\cftdotfill{\cftdotsep}}

%http://ctan.mirror.garr.it/mirrors/CTAN/macros/latex/contrib/appendix/appendix.pdf
\usepackage{appendix} %aggiunge dei comandi per l'appendice
\usepackage{parskip} %aiuta LaTeX a trovare il miglior stile per i page break
\setcounter{secnumdepth}{5} % numera i sottoparagrafi
\setcounter{tocdepth}{5} %aggiunge all'indice i sottoparagrafi
%\usepackage{titlesec} %\begin{paragraph} si può usare come subsubsubsection!

\usepackage{breakurl}%\url{...} può continare alla linea successiva. (si può andare a capo)

\usepackage[colorlinks=true]{hyperref}
\hypersetup{
	colorlinks=true,
	citecolor=black,
	filecolor=black,
	linkcolor=black, % colore dei link interni
	urlcolor=Maroon  % colore dei link interniesterni
}

%per alcune liste, permette di usare 'alligator' nei labeling 
\usepackage{blindtext} 
\usepackage{scrextend} 
\addtokomafont{labelinglabel} 
{\sffamily}
\begin{document}

\copertina
\tableofcontents
\newpage
\section{Introduzione}
Il centro estetico Nirvana vuole implementare un sito Internet al fine di poter fornire informazioni riguardo al centro stesso.\\
Il sito dovrà contenere informazioni riguardo i trattamenti disponibili e i macchinari utilizzati per essi, nonchè informazioni sulle consulenze e ogni informazione relativa a dove si trova il centro e quali orari di apertura osserva.\\
Inoltre, permette agli utenti di richiedere una consulenza o un trattamento, che necessita di essere confermato o meno dal centro stesso. Le prenotazioni possono anche inserite, oltre che confermate o smentite, anche dal centro stesso.\\
É fondamentale che il sito garantisca accessibilitá, in modo da permettere a chiunque di poter essere utilizzato, e usabilitá, separando struttura, presentazione e comportamento.\\
Si vuole infine garantire una navigazione fluida tra i contenuti del sito, evitando al piú possibile il disorientamento e prevedendo il giusto supporto per ritornare all'interno del sito stesso.\\
\section{Analisi}
\subsection{Studio dell'utenza finale}
Il sito vuole fornire informazioni riguardanti i possibili trattamenti che il centro estetico Nirvana offre, garantendo una navigazione fluida e con il minor numero di operazioni possibili.\\
Pertanto gli utilizzatori del sito sono visitatori casuali, clienti da breve o lungo tempo del centro e la responsabile del centro assieme ai propri dipendenti.
Si possono dunque distinguere tre categorie di utenti: l'utente generico, il cliente e i gestori del centro.\\
I clienti hanno il diritto di accedere ad aree riservate del sito, mentre i gestori possono anche accedere alle funzionalitá avanzate del sito. Entrambe le categorie possono essere definite come utenti interni.\\
L'utente finale é principalmente un utente posto tra l'utente generico e il cliente, ovvero un utente che a prescindere non conosce il linguaggio tecnico utilizzato, é dunque necessario che il sito abbia un linguaggio informale e semplice, in modo tale che sia comprensibile dalla maggior parte delle persone.
\subsection{Casi d'uso}
\subsubsection{Utente generico}
Un utente é definito \textit{generico} quando non é autenticato al sito, e pertanto puó solo visualizzare i servizi offerti dal centro, descritti dal sito stesso.\\
Dispone quindi dei seguenti casi d'uso:
\begin{itemize}
	\item Visualizzazione pagina "Home";
	\item Visualizzazione pagina "Trattamenti e Macchinari";
	\item Visualizzazione pagina "Trattamenti Viso";
	\item Visualizzazione pagina "Trattamenti Corpo";
	\item Visualizzazione pagina "Epilazione";
	\item Visualizzazione pagina "Massaggi";
	\item Visualizzazione pagina "Manicure \& Pedicure";
	\item Visualizzazione pagina "Macchinari";
	\item Visualizzazione pagina "Consulenze";
\end{itemize}
\paragraph{Visualizzazione pagina "Home"}\mbox{}\\
\label{par:VisHome}
L'utente generico può entrare nella pagina \textit{Home} in diversi modi:
\begin{itemize}
	\item se è appena entrato nel sito, è la prima pagina che viene visualizzata;
	\item se si trova in un'altra pagina, può raggiungere la homepage cliccando la scritta "Home" presente nella breadcrumb;
	%\item se si trova in un'altra pagina, può raggiungere la homepage cliccando sul logo presente nell'header;
\end{itemize}
All'interno di questa pagina l'utente può visualizzare una breve presentazione del centro estetico, oltre ad aver subito un piccolo menú dei servizi offerti.\\
\subsubsection{Cliente}
Un utente é identificato come \textit{cliente} quando é autenticato al sito e ha privilegi di utente, e pertanto puó accedere alla sua area personale ed effettuare delle richieste di prenotazione per i servizi offerti dal centro.\\
Per rispettare le regole di progetto, login e password per questo utente sono uguali a \textbf{user}.\\
Eredita dunque tutti gli use case di \textit{utente generico}, e dispone di casi d'uso extra:
\begin{itemize}
	\item Login "Area personale";
	\item Modifica password in "Area personale";
	\item Aggiunta richieste di prenotazione in "Gestione Prenotazione - Utente";
	\item Visualizzazione richieste di prenotazione in "Gestione Prenotazione - Utente";
	\item Logout "Area personale";
\end{itemize}
\subsubsection{Gestore}
Un utente é definito come \textit{gestore} quando é autenticato al sito e ha privilegi di amministratore, e pertanto puó accedere alla sua area personale ed inserire delle prenotazioni per i servizi offerti dal centro manualmente. Puó inoltre controllare le richieste attive e modificarle oppure confermarle o meno.\\
Sempre per rispettare le regole di progetto, login e password per questo utente sono uguali a \textbf{admin}.\\
Eredita dunque tutti gli use case di \textit{utente generico}, e dispone di casi d'uso extra:
\begin{itemize}
	\item Login "Area personale";
	\item Modifica password in "Area personale";
	\item Aggiunta prenotazione in "Gestione Prenotazione - Amministratore";
	\item Visualizzazioni prenotazioni in "Gestione Prenotazione - Amministratore";
	\item Modifica prenotazioni in "Gestione Prenotazione - Amministratore"
 	\item Eliminazione prenotazione in "Eliminazione Prenotazione - Amministratore"
	\item Logout "Area personale";
\end{itemize}
\section{Progettazione}
\section{Presentazione}
\section{Implementazione}
\section{Validazione}
\section{Fase di Test e Strumenti}
\section{Suddivisione del lavoro}
Per garantire una buona suddivisione del carico di lavoro che il progetto ha inevitabilmente richiesto, si é suddiviso il lavoro tra i membri del gruppo in questo modo:
\begin{itemize}
	\item Lisien Skenderi: 
	\begin{itemize}
		\item HTML: Sviluppo delle seguenti pagine: Index, Consulenze, Gestione Prenotazioni - Cliente;
		\item CSS: Modifiche relative al foglio per la visualizzazione desktop
		\item Javascript: Sviluppo delle funzioni per i menú
		\item PHP:
		\item Immagini:
		\item Relazione:
	\end{itemize}
	\item Matteo Cusin:
	\begin{itemize}
		\item HTML: Sviluppo delle seguenti pagine: Trattamenti Viso, Login e Registrazione, Eliminazione prenotazioni;
		\item CSS: Implementazioni sui fogli relativi alla visualizzazione desktop e mobile
		\item Javascript:
		\item PHP:
		\item Immagini:
		\item Relazione:
	\end{itemize} ;
	\item Annalisa Egidi:
	\begin{itemize}
		\item HTML: Sviluppo delle seguenti pagine: ;
		\item CSS:
		\item Javascript:
		\item PHP:
		\item Immagini:
		\item Relazione:
	\end{itemize} ;
	\item Nicola Baesso:
	\begin{itemize}
		\item HTML: Sviluppo delle seguenti pagine: ;
		\item CSS:
		\item Javascript:
		\item PHP:
		\item Immagini:
		\item Relazione:
	\end{itemize} ;
\end{itemize}

\end{document}
